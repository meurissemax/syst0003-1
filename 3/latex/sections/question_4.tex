\subsection{Constraints and simulations specifications}
The numerical values used for the simulations are identical to those used previously (homework 2).\par
The reference is set at $0$. It could possibly vary, but by a few centimetres at most.\par
The uncontrolled input signal is the wind. Its values have been determined previously (homework 2). The controlled input signal is the force applied to the damper mass to set it in motion. This force is between \num{0} and \SI{0}{\newton}.\par
The system consists of 4 states. The output is one of the states. In order to ensure that the behaviour of the system is physically realistic, we set a value domain for each state and will check in the simulations whether the values obtained belong to these domains.
\begin{table}[H]
    \centering
    \begin{tabular}{|l|c|}
        \hline
        {\bf State} & {\bf Domain}\\ \hline
        \hline
        $x_1 = d_1$ & ...\\ \hline
        $x_2 = \dot{d}_1$ & ...\\ \hline
        $x_3 = d_2$ & ...\\ \hline
        $x_4 = \dot{d}_2$ & ...\\ \hline
    \end{tabular}
    \caption{Range of acceptable values for each state}
\end{table}
