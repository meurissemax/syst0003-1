\subsection{State-space representation analysis}

% Stability
\subsubsection{Stability}
To study the stability of the system, we compute the eigenvalues of the dynamic matrix $A$ thanks to Matlab function (\texttt{eig}) :
\begin{align*}
    \lambda_1 &= \num{-5 + 8.6603i}\\
    \lambda_2 &= \num{-5 - 8.6603i}\\
    \lambda_3 &= \num{-0.0628 + 6.2828i}\\
    \lambda_4 &= \num{-0.0628 - 6.2828i}
\end{align*}
The system is stable if the real parts of the eigenvalue are all negative. In our case, the system is stable.

% Observability
\subsubsection{Observability}
To determine whether or not the system is observable, we compute the observability matrix thanks to Matlab function (\texttt{obsv}).\par
The matrix is full rank (verified with Matlab), the system is thus fully observable.\par
As seen on the matrix C, we need one sensor. According to the place of the non zero value, this sensor has to measure the $x_1$ state, namely the horizontal position of the top of the building $d_1$. This state is indeed the objective of the active mass damper and has thus to be observed.

% Controllability
\subsubsection{Controllability}
To determine whether or not the system is controllable, we compute the controllable matrix thanks to Matlab function (\texttt{ctrb}). In order not to take into account the uncontrollable input (wind), only the second column of the B matrix was kept for the calculation.\par
The matrix is full rank (verified with Matlab), the system is thus fully controllable.\par
As seen on matrix B, we need only one actuator. The first column of the $B$ matrix represents the wind, while the second one concerns the damper. This latter is indeed the only controllable input and contains two non-zero elements. As a result, only one actuator is needed, and acts on two states, the speed of the building and the speed of the damper, as they take place on $x_2$ and $x_4$.
