\subsection{Constraints and simulation specifications}
We have the following constraints : 
\begin{itemize}
    \item Acceleration of the mass damper between \num{0.3} and \num{0.6}$g$, as advised by Prof. Denoël.
    \item Power injected in the mass of below \SI{10}{\kilo\watt} so as to not have too much electrical consumption.
    \item Lateral movement of the top of the building not above \SI{1}{\meter}.
\end{itemize}
The scenario we look at is the following : 
A turbulent wind of maximum \SI{7.35}{\mega\newton}, that we represented as a sine function.
\subsubsection{Choice of cross-over frequency}
The frequency of our damper is computed via : $f = \sqrt{\dfrac{k}{m}} \approx \SI{10}{\hertz}$. We will therefore use a crossover frequency of \SI{20}{\hertz}, so all frequencies above that, probably coming from noise and unwanted phenomena, will be attenuated, while the amplitudes of the frequencies below that, which correspond to the internals of our system, will be amplified.
