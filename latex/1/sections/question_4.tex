\subsection{Control problem description}
\begin{itemize}
    \item {\bf Utility of the controller} : the controller (the algorithm) allows the system (the tower) to be active, {\it i.e.} to measure the oscillations to which it is subjected and to cancel it. Thanks to a servo-motor connected to the controller, the mass can move and reduce, or even eliminate totally, the oscillations.
    \item {\bf System to be controlled} : the tower (and the position of the tower is the signal)
    \item {\bf Inputs of the system} : wind force acting on the tower (uncontrollable) and force acting on the mass damper (controllable).
    \item {\bf Outputs of the system} : the relative distance between the vertical position and the displacement of the tower.
    \item {\bf Reference} : the vertical position of the tower.
    \item {\bf Actuators} : servo-motor to move the mass that reduces the oscillations.
    \item {\bf Constraints and limitations} : to simplify our system, we consider a tower \SI{200}{\meter} high, perfectly vertical when it undergoes no disturbance. The only disturbance on this tower is the strength of the wind. The wind, ranging from a few tens of \SI{}{\kilo\meter/\hour} to a hundred \SI{}{\kilo\meter/\hour}, can swing the tower from a few centimetres to several meters.
\end{itemize}
